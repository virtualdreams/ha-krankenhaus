\section{Fazit}
\label{sec:Fazit}

Gesundheitsförderung ist ein weites Feld und findet sich in den unterschiedlichsten Lebensbereichen wieder, sei es nun in der Schule, im Betrieb oder auch im Krankenhaus. Dem Setting Krankenhaus wird jedoch dank der Bestrebungen der WHO oder dem DNGfK zunehmend eine wichtigere Position in der Gesundheitsförderung für die Bevölkerung zuerkannt. Es steht zwar immer noch primär die medizinische Behandlung im Fokus (siehe Abb.2) jedoch bietet das Gesundheitsfördernde Krankenhaus viele Möglichkeiten um den Millionen Patienten pro Jahr ein Mehr an Gesundheitsgewinn zu bieten. 

Grundsatzdokumente wie die Ottawa-Charta oder den Homburger Leitlinien bieten konkrete Handlungsstrategien um eine gesundheitsförderliche Umwelt im Rahmen der Krankenhausversorgung zu gewährleisten.  Darüber hinaus könnten die Prinzipien bei praktischer Umsetzung zu einem gesellschaftlichen Umdenken in der Gesellschaft beitragen und so den aktuellen Entwicklungen Rechnung tragen, in denen der Menschen zum Koproduzent seiner Gesundheit befähigt werden soll. Dies setzt jedoch das Wissen jedes Einzelnen um Gesundheit und Gesundheitsförderung voraus. Dafür bedarf es mehr Forschung, staatliche Unterstützung und Öffentlichkeitsarbeit um diese Zielstellung zu erreichen. Nur so kann ein Bewusstsein dafür geschaffen werden, das Gesundheit von den Menschen in ihrer alltäglichen Umwelt geschaffen und gelebt wird, dort, wo sie spielen, lernen, arbeiten und lieben\footnote{vgl. Ottawa-Charta 1986, S. 4}. 

Die Verantwortung von Gesundheitsförderung obliegt jedoch meist noch dem medizinischen Institutionen und Mitarbeitern. Das Setting Krankenhaus kann, in seinem alltäglichen Umgang mit Gesundheit und Krankheit, Gesundheitsförderung als einen integralen Bestandteil der Krankenhauskultur vermitteln und zudem lebensverändernd wirken. Dieser Verantwortung sollte größte Unterstützung zuteilwerden, um Gesundheitsförderung als Menschenrecht in der Zukunft zu etablieren. 

Um die generelle Notwendigkeit von Gesundheitsförderung (auch im Setting Krankenhaus) zu bekräftigen, möchte ich mit folgendem Zitat meine Ausführungen abschließen.

\begin{quotation}
"`Da stehe ich am Ufer eines Flusses mit starker Strömung und höre den Hilferuf eines ertrinkenden Mannes. Ich springe sofort in den Fluss, lege meine Arme um ihn, schleppe ihn ans Ufer und führe eine künstliche Beatmung durch. Gerade als er wieder zu atmen beginnt, höre ich einen weiteren Hilferuf. Ich springe darauf wieder in den Fluss, erreiche den Mann, schleppe ihn ans Ufer, mache eine künstliche Beatmung und im gleichen Moment, als er wieder zu atmen beginnt, höre ich einen weiteren Hilferuf. Wieder zurück im Fluss geht das dann ständig so weiter. Ich bin so damit beschäftigt ins Wasser zu springen, die Männer ans Ufer zu schleppen und sie künstlich zu beatmen, das ich keine Zeit mehr habe danach zu schauen, wer um alles in der Welt alle diese Männer Fluss aufwärts ins Wasser wirft."'
\end{quotation}

\begin{flushright}
(Mc Kinlay 1979, In: Naidoo, Wills 2003)
\end{flushright}