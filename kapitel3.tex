\section{WHO-Konzept des Gesundheitsfördernden Krankenhauses (GFKH)}
\label{sec:WHOKonzeptDesGesundheitsförderndenKrankenhausesGFKH}

\subsection{Setting Krankenhaus}
\label{sec:SettingKrankenhaus}

Wie meinen Aussagen unter Punkt 2.2 zu entnehmen ist, ist die Gesundheitsförderung ein weites Betätigungsfeld, was mitnichten nur das Gesundheitssystem anspricht, sondern ebenso die Politik und das Gesundheitsbewusstsein eines jeden Menschen selbst. Trotzdem obliegt die Verantwortung für gesundheitsförderndes Verhalten oft dem medizinischen Personal, was neben der primären Versorgung beim Hausarzt insbesondere mit der  Krankenhausversorgung assoziiert wird. Doch ist prinzipiell nicht jedes Krankenhaus gesundheitsfördernd? Nähern wir uns dem Thema an.

Laut Statistischen Bundesamt gab es im Jahr 2012 in Deutschland 2017 Krankenhäuser die zusammen ca. 18,6 Millionen Patienten versorgten. In den Häusern waren zudem ca. 850.000 Menschen beschäftigt, darunter Ärzte, Pflegepersonal und weitere Angestellte\footnote{vgl. Statistisches Bundesamt 2012}. Das Krankenhaus stellt so ein spezifisches Setting dar, innerhalb derer Menschen Gesundheit schaffen und leben, in der Menschen arbeiten und lernen und in der Patienten ihre Krankheitsepisoden verarbeiten\footnote{vgl. BZgA 2011, S. 250}. Es ist ein Ort an dem ein erhöhtes Bewusstsein für Gesundheit und Krankheit existiert und an dem neue Wege bezgl. der eigenen Lebensweise eingeschlagen werden können. Das gesundheitsfördernde Krankenhaus soll jedoch nicht nur Verhaltensänderungen initiieren, sondern stellt vielmehr ein Setting dar, in dem Gesundheitsförderung ein integraler Bestandteil der Kultur des Krankenhauses ist\footnote{vgl. Naidoo, Wills 2003 S. 313f}. Das umfasst nicht nur die Gesundheit des Patienten, sondern auch die der Mitarbeiter, beinhaltet darüber hinaus aber auch gesundes Kantinenessen oder Kooperationsleistungen zu anderen Teams, Trägern etc.

Das Projekt "Gesundheitsförderndes Krankenhaus" basiert auf einer der fünf Handlungsstrategien der Ottawa-Charta und zwar "Die Gesundheitsdienste neu orientieren". 
Genaue Informationen und Leitlinien für die Umsetzung dieses Konzeptes können u. A. den Homburger Leitlinien (1999), der Budapest Deklaration (1991), den Wiener Empfehlungen (1997) oder der Chiemsee-Erklärung entnommen werden. Die wichtigsten Prinzipien werden innerhalb der Deklarationen jedoch wiederholt aufgegriffen, daher stütze ich mich im Folgenden auf die für mich schlüssigsten und  umfangreichsten Informationen, in Form der Homburger Leitlinien und der Budapester Deklaration. 

\subsection{Leitlinien und Prinzipien}
\label{sec:LeitlinienUndPrinzipien}

Definieren wir zunächst das Konzept Gesundheitsförderndes Krankenhaus (GFKH) genauer: 

\begin{quotation}
"Ein gesundheitsförderndes Krankenhaus leistet nicht nur eine qualitativ hochwertige umfassende medizinische und pflegerische Versorgung, sondern schafft auch eine die Ziele der Gesundheitsförderung verinnerlichende organisatorische Identität, baut eine gesundheitsförderliche Organisationsstruktur und –kultur auf, wozu auch die aktive, partizipatorische Rolle von Patienten und allen Mitarbeitern gehört, entwickelt sich zu einem gesundheitsförderlichen natürlichen Umfeld und arbeitet aktiv mit der Bevölkerung seines Einzugsgebietes zusammen."\footnote{WHO 1998 In: Homburger Leitlinien 1999 S. 1}
\end{quotation}

Krankenhäuser die Teil dieses Konzeptes werden möchten, akzeptieren die internationalen Normen und Standards der Patientenversorgung sowie der Gesundheitsförderung der Bevölkerung. Das Deutsche Netzwerk Gesundheitsfördernder Krankenhäuser (DNGfK) hat in diesem Zusammenhang im Rahmen der Homburger Leitlinien sechs Leitlinien konzipiert, deren Inhalt ich nun auszugsweise vorstellen möchte\footnote{Homburger Leitlinien 1999 S. 2ff)}. Zunächst ist hier der Aspekt des Gesundheitsgewinns zu nennen. GFKH sind stationäre Einrichtungen die den höchstmöglichen Gesundheitsgewinn für ihre Patienten erzielen möchten. Grundlage hierfür sind z.B. die Umsetzung von Pflegestandards oder auch die ganzheitliche Betrachtungsweise bei der Behandlung. Ziel ist die Förderung der Lebensqualität und die Befähigung zum selbstbestimmten Handeln (engl. Fachbegriff "Empowerment") seitens des Patienten. Der zweite zentrale Gesichtspunkt widmet sich der Patientenorientierung. Darunter versteht man nicht nur den gefühlvollen und freundlichen Umgang, unter Beachtung seiner Würde, sondern auch die Berücksichtigung von Patientenkarrieren und -perspektiven. Zudem soll der Patient zum Koproduzenten seiner Gesundung befähigt werden. Maßnahmen des GFKH sind aber nicht nur auf den Patienten ausgerichtet sondern auch auf die Mitarbeiter der Institution selbst. Das Krankenhaus soll eine Lebens- und Arbeitswelt schaffen, in der die Mitarbeiter ihre kooperativen und kommunikativen Kompetenzen ausbauen, selbstbestimmt handeln und so gesund arbeiten aber auch gesund bleiben können. GFKH bemühen sich weiterhin um Partnerschaften in der Region und nehmen gesundheitsförderlichen Einfluss auf die Bevölkerung. So könnte ein sogenanntes "Krankenhaus ohne Mauern" entstehen. Die fünfte Leitlinie widmet sich der Notwendigkeit von ökologischem Handeln. Ressourcenschonende und umweltfreundliche Initiativen stellen im Kontext der Gesunderhaltung von Lebensraum, einen wesentlichen Grundpfeiler gesundheitsförderlichen Verhaltens dar. Dieses füreinander "`Sorge tragen"' ist wesentlicher Bestandteil der Gesundheitsförderung an sich und wurde bereits unter Punkt 2.2 dargelegt. Die letzte Leitlinie betrachtet den wirtschaftlichen Faktor im Krankenhausbetrieb. Sie sind dazu angehalten mittels effizienter und kosteneffektiver Nutzung von Ressourcen, der ständigen Überprüfung von Angemessenheit, Nützlichkeit und Wirtschaftlichkeit von Leistungen oder auch durch Kooperationsleistungen zwischen niedergelassenen Ärzten ein wirtschaftliches Umdenken zu entwickeln und eine Transparenz der Leistungen ermöglichen.
Ergänzen möchte ich diese Aussagen noch durch einige Zielstellungen der Budapester Erklärung. Zu nennen wäre hier insbesondere die Verbesserung der Ernährungsangebote für Mitarbeiter und Patienten, regelmäßige personelle Trainings- und Ausbildungsprogramme aber auch das Schaffen von aktivierenden Lebensumfeldern für Chronisch Kranke oder Langzeitpatienten.

Nachdem nun die Grundlagen des Konzeptes zumindest im Ansatz dargelegt wurden, möchte ich mich nun der bereits angesprochenen deutschen Umsetzung dieses Vorhabens widmen, in Form des Deutschen Netzwerkes Gesundheitsfördernder Krankenhäuser (DNGfK).

\subsection{Deutsche Netzwerk Gesundheitsfördernder Krankenhäuser (DNGfK)}
\label{sec:DeutscheNetzwerkGesundheitsfördernderKrankenhäuserDNGfK}

Das DNGfK entwickelte sich im Zuge des Pilotprojektes "`Gesundheitsförderndes Krankenhaus"' (oder engl: Health Promoting Hospitals, HPH) der WHO im Jahre 1993. Damals nahmen 20 Krankenhäuser aus 11 europäischen Ländern, davon 5aus Deutschland, an diesem Projekt teil. Heute gibt es bereits 800 teilnehmende Krankenhäuser, die in 23 europäischen Ländern organisiert sind\footnote{vgl. DNGfK 2014}.

Die Gründung des deutschen Netzwerkes vollzog sich schließlich im Jahr 1996 in Form eines Vereins. Mit den Ausarbeitungen der Homburger Leitlinien und der Chiemsee-Erklärung präsentierte sich die Vereinigung anschließend klar in der Öffentlichkeit\footnote{vgl. BZgA 2011, S. 252}. Wie viele Krankenhäuser (und Rehabilitationseinrichtungen) mittlerweile zum Netzwerk gehören, ist nicht eindeutig zu klären. In der neueren Literatur werden zwischen 58-65 Institutionen angegeben, bei der näheren Recherche auf der Homepage des Vereins konnten nur 52 beteiligte anerkannte Einrichtungen und ordentliche Mitglieder herangezogen werden. Auffällig war der hohe Anteil von Kliniken in Nordrheinwestfalen mit 19. In Sachsen nehmen im Jahr 2014 drei Kliniken an diesem Projekt teil, u. A. das Städtische Krankenhaus Dresden-Neustadt, welches später nochmal aufgegriffen wird.

Das DNGfK zeichnet sich durch einen ansprechenden Internetauftritt aus, ist übersichtlich und strukturiert aufbereitet und verhilft einem so zu einem schnellen überblickartigen Wissenszugang von der Grundidee oder der Geschichte des Vereins bis hin zu den Grundsatzdokumenten der Vereinigung.

Doch was für gesundheitsfördernde Aktionen kann ich mir unter Beachtung der Leitlinien nun konkret in deutschen Krankenhäusern vorstellen? Dazu nun ein praktischer Exkurs.

\subsection{Beispielhafte Darstellung der Konzept-Umsetzung }
\label{sec:BeispielhafteDarstellungDerKonzeptUmsetzung}

\subsubsection{Zielstellung}
\label{sec:Zielstellung}

Wie bereits erwähnt, verpflichten sich die Mitglieder des DNGfK dazu die Richtlinien der Grundsatzdokumente wie z.B. die Ottawa-Charta oder die Budapester-Erklärung, aber auch konkrete Leitlinien in Form der Homburger Leitlinien anzunehmen und umzusetzen. Diese Empfehlungen orientieren sich am Gesundheitsgewinn, den  Patienten,  Mitarbeitern und Partnerschaften, aber auch an Ökologischen und wirtschaftlichen Gesichtspunkten. Wie diese Umsetzung erfolgt, kann jedoch von Krankenhaus zu Krankenhaus verschieden sein und bedarf konkreterer Informationen. Anhand von zwei Krankenhäusern soll die Ausgestaltung gesundheitsfördernder Maßnahmen nun exemplarisch dargelegt werden. 

\subsubsection{Städtisches Krankenhaus Dresden-Neustadt}
\label{sec:StädtischesKrankenhausDresdenNeustadt}

Das Städtische Krankenhaus Dresden-Neustadt ist eines von drei sächsischen gesundheitsfördernden Krankenhäusern und ist zudem akademisches Lehrkrankenhaus der Technischen Universität Dresden. Das Haus bietet interessierten Mitbürgern eine umfangreiche gestaltete Homepage, bei der Informationsrecherche zum Thema Gesundheitsförderung wird man jedoch enttäuscht. Es wird zwar auf die inhaltliche Ausrichtung nach den Richtlinien im Sinne der Ottawa-Charta oder auch der Homburger Leitlinien verwiesen, detaillierte Informationen hingegen fehlen. Zur Umsetzung von Gesundheitsförderung werden nur folgende Projekte benannt: Erarbeitung und Anwendung von Pflegestandards; ökologische Projekte; Mitarbeiter- und Patientenbefragung; Stressbewältigungstraining; Rückenschule; Langzeitprojekt "`Kulturelle Betreuung"'\footnote{vgl. Homepage Städtisches Krankenhaus Dresden-Neustadt 2014}. Was sich nun konkret hinter diesen Projekten verbirgt bzw. wie diese im Detail umgesetzt werden, ist für Außenstehende nur schwer zu beurteilen. 

\subsubsection{AMEOS Klinikum Staßfurt}
\label{sec:AMEOSKlinikumStassfurt}

Dahingegen ist die gesundheitsförderliche Politik in der AMEOS Klinik in Staßfurt bedeutend detaillierter dargelegt. Das Klinikum ist seit 2007 Mitglied des DNGfK und verweist auf seiner Homepage auf einige interessante Projekte zum Thema, wovon einige nun kurz vorgestellt werden. Das Projekt "`Rauchfrei"' (Rauchfreies Krankenhaus) wird als selbstverständliche Aktion zum Schutz für Patienten und Mitarbeiter angesehen, da Menschen im Krankenhaus nicht nur genesen, sondern auch zu einer gesundheitsförderlichen Lebensweise befähigt werden sollen. Weiterhin initiiert das Haus Partnerschaften und gemeinsame Projekte zu benachbarten Schulen zum Thema Gesundheit und Schule. Aber auch Bedürfnisse der Patienten und Mitarbeiter werden berücksichtigt. Mitarbeiter haben die Möglichkeit der Teilnahme an Entspannungs- und Rückenschulkursen um so physische und psychische Belastungssituationen abzumildern und mögliche Ausfallzeiten zu minimieren. Für chronisch kranke Patienten hingegen besteht die Möglichkeit des "`Wundmanagements"'. Sobald eine Problemwunde auftritt, wird der Wundmanager gerufen der sich um die weitere individuell ausgerichtete Therapie sorgt\footnote{vgl. Homepage Ameos-Klinikum 2014}.

Diese Aktionen werden zwar hinreichend dargelegt, bedürften aber zur Beurteilung von Qualität und Umsetzung wesentlich mehr Informationen. Die Differenzierung von (vielversprechender) Theorie und reeller Praxis ist auch hier wieder für Außenstehende schwer nachzuvollziehen.

\begin{figure}[h]
	\centering
		\includegraphics[scale=2.75]{rauchfreiymbol.jpg}
	\caption{Rauchfreisymbol der AMEOS-Klinik. Homepage der AMEOS-Klinik 2014}
	\label{fig:rauchfreiymbol}
\end{figure}

\subsubsection{Perspektive}
\label{sec:Perspektive}

Mit Verabschiedung der Ottawa-Charta und dem damit einhergehenden Aufruf zur "`Gesundheit für alle bis zum 2000"' und darüber hinaus) ist das Thema Gesundheit und Gesundheitsförderung zunehmend in den Blickpunkt der weltweiten Öffentlichkeit gerückt. Im Zuge dessen wurden viele Gesundheitsförderungsresolutionen unterschrieben, denen jedoch nur begrenzt konkrete Maßnahmen folgten\footnote{vgl. Bangkok-Charta 2005, S. 6}. Die Forderung zur Neuausrichtung der Gesundheitsdienste führte schließlich zur Gründung des DNGfK 1996. Das deutsche Netzwerk strebt "`Gesundheit für alle"' unter Anerkennung der dortigen Leitlinien an. Auch wenn sich dieses Netzwerk seit seinem Bestehen konzeptionell und institutionell konsolidiert, qualitativ verbessert und quantitativ kontinuierlich gewachsen ist, haben sich die hohen Erwartungen an das Netzwerk nur begrenzt erfüllt\footnote{vgl. BZgA 2011, S. 253}. Dennoch scheint die Integration von gesundheitsfördernden Strukturen in den Krankenhäusern geglückt zu sein, da ein Großteil der gestarteten Projekte längerfristigen Bestand hat und so ein Nutzen für das Krankenhaus vorhanden sein muss\footnote{vgl. Trojan, Legewie 2001 In: Waller 2002, S. 174}. Dem gegenüber steht die Aussage der Bundeszentrale für gesundheitliche Aufklärung: "`Nur ein Bruchteil aller Krankenhäuser macht im Netzwerk mit, und auch die Häuser die mitmachen, sind zumeist weit von der radikalen Neuorientierung entfernt, wie sie durch die Ottawa-Charta angestrebt wurde."'\footnote{BZgA 2011, S. 253}

Aufgrund des hohen Wettbewerbsdrucks sehen sich zudem heute viele Krankenhäuser in ihrer Existenz bedroht. Hinzukommt die zunehmende physische und psychische Belastung des Personals, aber auch Sorge um Qualitätseinbußen. All diese Faktoren lassen viele Verantwortliche und Mitarbeiter glauben das die innovativen Projekte eines gesundheitsfördernden Krankenhauses nur eine zusätzliche personelle und finanzielle Belastung für das routinierte Kerngeschäft bedeutet\footnote{vgl. Homburger Leitlinien 1999, S. 6}. Um diesen Irrglauben entgegen zu wirken, bedarf es Forschung. Es ist anzustreben, das Konzept des GFKH mit dem Aspekt von Wirtschaftlichkeit und Qualitätsverbesserung zu assoziieren um so einen Wettbewerbsvorteil zu erlangen und gleichzeitig die Gesundheitsförderung zu etablieren.

Man kann somit festhalten, dass die gewollte Ausdehnung gesundheitsfördernder Krankenhäuser und Aktionen zwar noch am Anfang steht, jedoch auch erste Erfolge in der Umsetzung zu verzeichnen sind. Unzureichende gesundheitspolitische Förderung und Unterstützung sind eine der Hürden denen sich das DNGfK derzeit und zukünftig stellen muss. Ein politisches und gesellschaftliches Umdenken könnte den Stellenwert der Gesundheitsförderung und damit das Konzept des gesundheitsfördernden Krankenhauses in Richtung des SOLL-Wertes von Hurrelmann künftig erhöhen (siehe Abb. SOLL-Zustand der Gesundheitsförderung; nach Hurrelmann et al 2004, S. 17).

\begin{figure}[h]
	\centering
		\includegraphics{ist-soll.png}
	\caption{SOLL-Zustand der Gesundheitsförderung. Frei nach Hurrelmann et al}
	\label{fig:ist-soll}
\end{figure}