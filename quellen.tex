\begin{thebibliography}{99}

	\bibitem{AMEOS-Klinik}AMEOS-Klinik (2014): Gesundheitsförderung. \url{http://www.ameos.eu/5119.html/}.

	\bibitem{Bangkok Charta}Bangkok Charta (2005): Gesundheitsförderung. \url{http://dngfk.de/downloads/}.
	
	\bibitem{Budapester Deklaration}Budapester Deklaration (1991): Gesundheitsförderndes Krankenhaus. \url{http://dngfk.de/downloads/}.
	
	\bibitem{Bundeszentrale für gesundheitliche Aufklärung}Bundeszentrale für gesundheitliche Aufklärung (Hrsg.) (2011): Leitbegriffe der Gesundheitsförderung und Prävention. Glossar zu Konzepten, Strategien und Methoden. Gamburg: Verlag für Gesundheitsförderung.
	
	\bibitem{Deutsches Netzwerk gesundheitsfördernder Krankenhäuser}Deutsches Netzwerk gesundheitsfördernder Krankenhäuser (DNGfK) (2014): Homepage des DNGfK. \url{http://dngfk.de/}.
	
	\bibitem{Haisch, Hurrelmann, Klotz}Haisch, Jochen; Hurrelmann, Klaus; Klotz, Theodor (Hrsg.)(2006): Medizinische Prävention und Gesundheitsförderung. Bern: Verlag Hans Huber.
	
	\bibitem{Homburger Leitlinien}Homburger Leitlinien (1999): Gesundheitsförderndes Krankenhaus. \url{http://dngfk.de/downloads/}.
	
	\bibitem{Hurrelmann, Klotz, Haisch}Hurrelmann, Klaus; Klotz, Theodor; Haisch, Jochen (2004): Lehrbuch Prävention und Gesundheitsförderung. Bern: Verlag Hans Huber.
	
	\bibitem{Hurrelmann, Laaser}Hurrelmann, Klaus; Laaser, Ulrich (1993): Gesundheitswissenschaften. Handbuch für Lehre, Forschung und Praxis. Weinheim und Basel: Beltz Verlag.

	\bibitem{Naidoo, Wills}Naidoo, Jennie; Wills Jane; Bundeszentrale für gesundheitliche Aufklärung (Hrsg.) (2003): Lehrbuch der Gesundheitsförderung. Umfassend und anschaulich mit vielen Beispielen und Projekten aus der Praxis der Gesundheitsförderung. Gamburg: Verlag für Gesundheitsförderung.
	
	\bibitem{Österreichisches Netzwerk Gesundheitsfördernder Krankenhäuser}Österreichisches Netzwerk Gesundheitsfördernder Krankenhäuser (2000): Abbildung des WHO-Schirms zur Gesundheitsförderung. \url{http://oengk.univie.ac.at/downloads/rb-alt/rb9.html/}.
	
	\bibitem{Ottawa-Charta}Ottawa-Charta (1986): Grundlagen der Gesundheitsförderung. \url{http://dngfk.de/downloads/}.
	
	\bibitem{Statistisches Bundesamt}Statistisches Bundesamt (2012): Anzahl Krankenhäuser in Deutschland. \url{http://www.destatis.de/DE/ZahlenFakten/GesellschaftStaat/Gesundheit/Krankenhaeuser/Tabellen/KrankenhaeuserJahreVeraenderung.html/}.
	
	\bibitem{Städtisches Krankenhaus Dresden-Neustadt}Städtisches Krankenhaus Dresden-Neustadt (2014): Gesundheitsförderung. \url{http://www.khdn.de/gesundheitsfoerderndes-krankenhaus,42.php/}.
	
	\bibitem{Waller}Waller, Heiko (2002): Gesundheitswissenschaft. Eine Einführung in Grundlagen und Praxis von Public Health. 3.Auflage. Stuttgart: Verlag W. Kohlhammer.
\end{thebibliography}
