\begin{thebibliography}{99}

	\bibitem{Backes, Clemens} Backes, Gertrud M.; Clemens, Wolfgang (1998): Lebensphase Alter. Eine Einführung in die sozialwissenschaftliche Alternsforschung. Weinheim; München: Juventa Verlag.
	
	\bibitem{Brosseder} Brosseder, Johannes et al. (2011): Forum Systematik. Bd. 40: Wenn das Leben unerträglich wird. Suizid als philosophische und pastorale Herausforderung, hrsg. von Emmanuel J. Bauer et al. Stuttgart: Verlag W. Kohlhammer.
	
	\bibitem{Colla-Müller, Herbert} Colla-Müller, Herbert E. et al. (1994): Niedersächsische Beiträge zur Sozialpädagogik und Sozialarbeit. Bd. 13: Konfrontation der Sozialpädagogik mit Sterben und Tod. Entwurf eines sozialpädagogisch verantworteten Umgangs mit tödlich Erkrankten und Sterbenden, hrsg. Von Karl-Heinz Karusseit. Frankfurt am Main: Europäischer Verlag der Wissenschaften.
	
	\bibitem{Eink, Haltenhof} Eink, Michael; Haltenhof, Horst (2006): Basiswissen: Umgang mit suizidgefährdeten Menschen. Bonn: Psychiatrie-Verlag.
	
	\bibitem{Erben} Erben, Christina (2010): Sterbekultur im Krankenhaus und Krebs. Handlungsmöglichkeiten und Grenzen sozialer Arbeit. 2.Auflage.Oldenburg: Paulo Freire Verlag.
	
	\bibitem{Erlemeier} Erlemeier, Norbert (2011): Suizidalität und Suizidprävention im höheren Lebensalter. Stuttgart: Verlag W. Kohlhammer. 
	
	\bibitem{Fachhochschule Köln} Fachhochschule Köln (2012): Klaus Mathuse Soziale Arbeit in der Gerontopsychiatrie. \url{http://www.sw.fh-koeln.de/groups/diesacademicus2012/} - Download vom 19.08.2013.
	
	\bibitem{Feldmann} Feldmann, Klaus (1997): Sterben und Tod. Sozialwissenschaftliche Theorien und Forschungsergebnisse. Opladen: Leske + Budrich.
	
	\bibitem{Forschungsgruppe KomPASS der Fachhochschule Bielefeld} Forschungsgruppe KomPASS der Fachhochschule Bielefeld (2010): komPASS Kompetenzentwicklung im Gesundheits- und Sozialbereich. Bd. 1: Kompetenz und Kooperation im Gesundheits- und Sozialbereich, hrsg. von Ursula Walkenhorst et al. Berlin: LIT Verlag.
	
	\bibitem{Gangelter Einrichtungen Maria Hilf} Gangelter Einrichtungen Maria Hilf. Gerontopsychiatrische Station. \url{http://www.gangelter-einrichtungen.de/linkes-menue/krankenhaus-maria-hilf/stationaere-angebote/gerontopsychiatrische-station.html} - Download vom 12.08.2013.
	
	\bibitem{Kirch} Kirch, W.; Vodenitscharov, S.; Krappweis, H.(Hrsg.) (1998): Geriatrische Rehabilitation. Regensburg: S. Roderer Verlag.
	
	\bibitem{Krueger, Golz} Krueger, Joachim; Golz, Anita (1995): Fontane, Theodor: Gedichte. Bd. 2. Einzelpublikationen, Gedichte in Prosatexten, Gedichte aus dem Nachlaß. 2. Auflage. Berlin:  Aufbau-Verlag, S. 464.
	
	\bibitem{Müller} Müller, Stephan E. (2010): Assistierter Tod. Ethische und theologische Perspektiven. In: Müller, Stephan E.; Mäde, Erwin: Glaube und Ethos. Theologie im interdisziplinären Dialog. Bd. 9: Menschenwürdig sterben - aber wie?.  Medizinische, juristische und ethische Aspekte, hrsg. von Rainer Beckmann, Stephan E. Müller; Berlin: LIT Verlag, S. 115-136.
	
	\bibitem{Peplau} Peplau, Hildegard E. (1997): Zwischenmenschliche Beziehungen in der Pflege - Ausgewählte Werke. Bern: Verlag Hans Huber.
	
	\bibitem{PQSG1} PQSG - Portal für Qualitätsmanagement und Service in der geriatrischen Pflege (2008): Stellenbeschreibung für eine Pflegefachkraft Schwerpunkt Gerontopsychiatrie. \url{http://www.pqsg.de/seiten/openpqsg/hintergrund-stellenbeschreibung-pk-geronto.htm} - Download vom 20.08.2013.
	
\end{thebibliography}
