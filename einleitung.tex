\section{Einleitung}
\label{sec:Einleitung}

Soziale Gesellschaften befinden sich stets in einem fortwährenden systemübergreifenden
und strukturellen Wandel. Aufgrund flächendeckender medizinischer Versorgung, der Installation sanitärer Anlagen oder der Verfügbarkeit von sauberem Trinkwasser haben sich die Lebensbedingungen der Menschen in den letzten Jahren und Jahrzehnten sehr zum Positiven entwickelt und so zu einer gestiegenen Lebenserwartung beigetragen. Parallel zum technischen und medizinischen Fortschritt hat sich aber auch das Verständnis von Gesundheit und Krankheit in der Medizin und in der Bevölkerung im Laufe der Zeit gewandelt. Es kommt zunehmend zu einer Abwendung von der bloßen Krankheitsfokussierung hin zur gesundheitserhaltenden bzw. gesundheitsförderlichen Behandlung in der Medizin. Trotzdem wird Gesundheit auch heute noch oft als bloße Abwesenheit von Krankheit angesehen.

Unterstützt durch frei zugängliche internetbasierte Informationen und den medizinischen Fortschritt können Menschen heute ihre persönliche Lebensweise im Kontext von Gesundheit und Krankheit, bedeutend differenzierter verstehen und beeinflussen. Im Zuge dieser Entwicklung sieht sich das Gesundheitswesen, gleichzeitig konfrontiert mit der Zunahme chronischer Erkrankungen, steigender medizinischer Kosten, Stellenrationalisierungen und institutionellem Konkurrenzdruck, vor ganz neue Herausforderungen gestellt. Einige Krankenhäuser trachten daher nach qualitätssteigernden Faktoren die ihre Position in der Umgebung etablieren und festigen. 

Bietet das Konzept der World Health Organziation (WHO) eines gesundheitsfördernden Krankenhauses in diesem Zusammenhang eine Option dieser gesundheitlichen Wendung Rechnung zu tragen, indem es zugleich Menschen verstärkt für ihre Lebensweise sensibilisiert und die eigene institutionelle Existenz sichert?

Um sich diesem Thema vertieft zu widmen, bedarf es vorab einer grundlegenden Darstellung von Gesundheitsförderung an sich. Das dargelegte Wissen um Gesundheitsförderung dient anschließend als theoretische Basis für die darauf folgenden Ausführungen. Unter Punkt zwei wird daher der Begriff definiert, dessen Ziele und Prinzipien vorgestellt, untermauert von Einblicken in die Gesundheitsförderung in Deutschland und im Medizinstudium. Darauf aufbauend folgt das Konzept des gesundheitsfördernden Krankenhauses, als Teilaspekt von Gesundheitsförderung, unter Punkt drei. Hier soll zunächst das Setting Krankenhaus sowie Ziele und Prinzipien dieses Konzeptes vorgestellt werden. Im Weiteren wird die deutsche WHO-Umsetzung in Form des Deutschen Netzwerkes Gesundheitsfördernder Krankenhäuser (DNGfK) mittels der beispielhaften Vorstellung einzelner Institutionen vorgestellt. Zuletzt erfolgt die perspektivische Auslegung des Konzeptes im Kontext von Vision und Realität. 

Bei der Recherche habe ich mich primär auf die umfangreiche Literatur der Bibliothek der Technischen Universität Dresden zum Thema "`Gesundheitsförderung"' und "`Gesundheitsfördernder Krankenhäuser"' gestützt, aber auch die Literaturempfehlungen der Homepage des Deutschen Netzes Gesundheitsfördernder Krankenhäuser und Gesundheitseinrichtungen e.V. genutzt.
 
Ziel der Arbeit ist die Vorstellung des WHO-Konzeptes "`Gesundheitsfördernder Krankenhäuser"', unter Beachtung der Einordnung in das Thema Gesundheitsförderung. Ich möchte in diesem Zusammenhang neben theoretischen Grundlagen wie der Ottawa-Charta auch konkrete Handlungsstrategien in einem solchen Krankenhaus exemplarisch vorstellen, um dieses theoretische Konzept so praktisch handhabbar zu machen. Kann dieses Konzept wirklich zu einer gesundheitlichen Neuausrichtung im alltäglichen Krankenhausbetrieb beitragen?

Ich gebe jedoch zu beachten, dass die folgenden Ausführungen keinen Anspruch auf Vollständigkeit besitzen und daher keine qualifizierte Beurteilung des Konzeptes ermöglichen. Im Rahmen dieser Arbeit können viele Thematiken nur kurz bzw. exemplarisch behandelt werden. 

